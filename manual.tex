%% LyX 2.3.4 created this file.  For more info, see http://www.lyx.org/.
%% Do not edit unless you really know what you are doing.
\documentclass{scrartcl}
\usepackage{fontspec}
\setmainfont[Ligatures=TeX,Numbers=OldStyle]{Metropolis}
\setsansfont[Ligatures=TeX]{Metropolis Extra Bold}
\setmonofont{Metropolis Medium}
\usepackage[a4paper]{geometry}
\geometry{verbose,tmargin=2cm,bmargin=3cm,lmargin=2.5cm,rmargin=2.5cm,footskip=2cm}
\pagestyle{plain}
\setcounter{secnumdepth}{2}
\setcounter{tocdepth}{2}
\setlength{\parskip}{\medskipamount}
\setlength{\parindent}{0pt}
\usepackage{color}
\usepackage{url}
\usepackage{graphicx}
\usepackage{microtype}
\usepackage[unicode=true,
 bookmarks=true,bookmarksnumbered=false,bookmarksopen=true,bookmarksopenlevel=1,
 breaklinks=false,pdfborder={0 0 1},backref=false,colorlinks=true]
 {hyperref}
\hypersetup{pdftitle={Digit Manual},
 pdfauthor={Poly Effects},
 pdfsubject={Poly Digit Manual},
 pdfkeywords={digt},
 linkcolor=pink,citecolor=pink}

\makeatletter
\@ifundefined{date}{}{\date{}}
%%%%%%%%%%%%%%%%%%%%%%%%%%%%%% User specified LaTeX commands.
\addtokomafont{section}{\MakeUppercase}
\addtokomafont{subsection}{\MakeUppercase}
\usepackage{fontspec, newunicodechar}
\newfontfamily{\fallbackfont}{Arial}[Scale=MatchLowercase]
\DeclareTextFontCommand{\textfallback}{\fallbackfont}
\newunicodechar{¼}{\textfallback{¼}}
\newunicodechar{⅓}{\textfallback{⅓}}
\newunicodechar{⅛}{\textfallback{⅛}}
\definecolor{pink}{RGB}{255, 117, 208}

\makeatother

\begin{document}
\begin{center}
\includegraphics[width=0.9\columnwidth]{D:/git_repos/polydigit/poly_promo_assets/Poly/FB-Cover-image-4-828x465px}
\par\end{center}

I'm currently rewriting the manual to include the new Beebo modules!
If you're ever confused by anything or need help please contact me.
This manual is a compliment to the videos. Please watch the videos
at \url{https://www.youtube.com/channel/UCwQ9E6imYd0qFPMEuEZjcdw}
\begin{quote}
Manual is currently a work in progress!

\newpage{}
\end{quote}
\tableofcontents{}

\newpage{}

\section{Getting Started }

\subsection{Power}

The first thing to do is power the pedal. It's 9V center negative
and requires 500mA. After that it'll start up. Start up time is currently
a bit long, so don't panic if nothing is happening for a few seconds.

Once Digit / Beebo has started up we're at the main screen, with the
default preset loaded. From here we can add modules, change the settings
of existing modules, connect modules or load a preset. This leads
to the question, what is a module?
\begin{center}
\includegraphics[width=0.9\textwidth]{\string"D:/photos/Digit_screenshots/design/2020-06-26 17_23_18-Digit 2\string".png}
\par\end{center}

\subsection{Modules}

Modules are the basic blocks you connect. Think of them as individual
effect pedals on a pedal board that you place and connect. There are
some modules that process audio, and some modules that you can use
to control other modules. The ability to control modules is the powerful
modular / eurorack style workflow that differentiates Digit / Beebo
from many other systems.

\subsection{Add a module}

To start, we'll add a delay to the default preset. To do this, tap
plus and choose delay. 
\begin{center}
\includegraphics[width=0.9\textwidth]{\string"D:/photos/Digit_screenshots/design/2020-06-26 17_24_12-Digit 2\string".png}
\par\end{center}

The new delay will appear on the screen and you can now drag it to
a comfortable spot. You can drag any module at any time to move it.
\begin{center}
\includegraphics[width=0.9\textwidth]{\string"D:/photos/Digit_screenshots/design/2020-06-26 17_24_43-Digit 2\string".png}
\par\end{center}

To start I recommend tapping the help button (the ? mark) which will
turn on the extra help labels.

\subsection{Connect a module}

You'll now want to connect up the delay. Hold input 1, then with another
finger tap the delay you've added. 
\begin{center}
\includegraphics[width=0.9\textwidth]{\string"D:/photos/Digit_screenshots/screenshots/2020-02-10 19_28_38-Digit 2\string".png}
\par\end{center}

This is just the same as connecting up pedals on a pedal board, but
you can split and merge signals easily. You choose a source and then
a module to connect to. So now hold delay 1 and then tap output 1
to connect them. If you're connecting things with multiple inputs
or outputs a screen will appear where you can select which to use. 
\begin{center}
\includegraphics[width=0.9\textwidth]{\string"D:/photos/Digit_screenshots/design/2020-06-26 17_25_35-Digit 2\string".png}
\par\end{center}

\subsection{Change settings }

We can now change the settings of the delay. Tap on it to get up the
controls. Many modules have extra controls in the side menus.
\begin{center}
\includegraphics[width=0.9\textwidth]{\string"D:/photos/Digit_screenshots/design/2020-06-26 17_26_07-Digit 2\string".png}
\par\end{center}

\subsection{Knobs}

Once you touch any slider to change it, it automatically maps to the
knobs. The right knob changes the value a lot (coarse) and the left
by a little (fine).

\subsection{Presets}

To save or load a preset tap the floppy disk icon. You'll also see
a set list option. This is what order you move through the presets
in your set list when pressing both center and right or center and
left foot switches together.

\subsection{Foot switches}

By default, the right most foot switch is bypass and the two others
don't do anything. Foot switches can act as tempo sources or as control
sources. To connect a foot switch to something, first tap add (+)
and then select foot switch A or B. A is the left one, B is the center,
C is the right.
\begin{center}
\includegraphics[width=0.9\textwidth]{\string"D:/photos/Digit_screenshots/design/2020-06-26 17_33_59-Digit 2\string".png}
\par\end{center}

Now tap the foot switch module you've added and then tap connect.
You can now connect it to things you want to control. There are two
outputs, the tap tempo and its value. Value is if it's up or pressed
down. To connect tap tempo from the foot switch, hold the foot swtich
and tap the delay. Then choose the BPM out.
\begin{center}
\includegraphics[width=0.9\textwidth]{\string"D:/photos/Digit_screenshots/design/2020-06-26 17_34_26-Digit 2\string".png}
\par\end{center}

\begin{center}
\includegraphics[width=0.9\textwidth]{\string"D:/photos/Digit_screenshots/design/2020-06-26 17_35_03-Digit 2\string".png}
\par\end{center}

\subsection{LFO / control signals}

As you saw with foot switches, one module can control another. If
we now add a low frequency oscillator (LFO) we can use that to control
other settings on the delay. Add the LFO then hold it and tap the
delay. Then choose warp in the screen that appears.
\begin{center}
\includegraphics[width=0.9\textwidth]{\string"D:/photos/Digit_screenshots/design/2020-06-26 17_34_43-Digit 2\string".png}
\par\end{center}

Control signals are coloured green.
\begin{center}
\includegraphics[width=0.9\textwidth]{\string"D:/photos/Digit_screenshots/design/2020-06-26 17_36_08-Digit 2\string".png}
\par\end{center}

The effect will be quite strong, so if we want to reduce we can tap
on the LFO and reduce the level. A uni polar LFO will only got from
0-1 (positive) but if you set that to zero it'll go negative and positive.
\begin{center}
\includegraphics[width=0.9\textwidth]{\string"D:/photos/Digit_screenshots/design/2020-06-26 17_55_25-Digit 2\string".png}
\par\end{center}

We can also add other modules in between such as the attenuverter
that act on the control signals. We can invert the phase or reduce
the level with this as well, in case we want the same LFO to control
two different things at different amounts or phase.
\begin{center}
\includegraphics[width=0.9\textwidth]{\string"D:/photos/Digit_screenshots/design/2020-06-26 17_36_50-Digit 2\string".png}
\par\end{center}

The attenuverter attenuates and/or inverts a signal. It has two inputs
which can be either be controlled by the sliders or by a control input.
You can attenuate / invert one control signal with another allowing
you ring mod / AM style control possibilities.

\subsection{Firmware update}

If you currently have version 1.something (eg 1.7) firmware installed
(check in the setting menu) get all the packages in the 'upgrade\_from\_1'
folder from here and put them on a usb flash drive. Tap upgrade firmware
and wait while nothing appears to happen for about 20 minutes. When
you press the firmware update button there's no feedback and nothing
happens at all, it is working but the early firmware update interface
is pretty terrible. Just walk off and come back in a bit. The pedal
should automatically restart with the new firmware. If it hasn't restarted
after 35 minutes, panic and contact me. It takes quite a while. Updates
in the 2 series only take a minute or two. 

The text that comes up at the end will have many messages and one
of the messages will say : 'Failed to unmount media.' This is okay.
Don’t turn it off then. It’ll will restart itself soon

A full video on how to copy the files and update is available here: 

\url{https://youtu.be/5fXnT4UiqGQ}

If you have version 2 installed, please grab just the 2 packages in
the 'upgrade\_from\_2' folder from here and put them on a usb flash
drive and click upgrade. It should be very quick. It'll restart itself.

The USB flash drive must be formatted FAT32 and have one partition
on it. If the update fails, please try pressing export presets in
the settings screen. If that also fails then the USB flash drive is
unreadable, either due to having no partitions (partitonless drive)
or being the wrong format. Please contact me for help. Please use
a normal USB flash drive and not an external hard drive. The cheap
NXT ones from Staples work well. All Sandisk ones tested also worked
well. Verbatim ones tested have failed.

\subsection{Inputs as effects loops}

You can use the input / outputs of Digit as effect loops. For example,
physically connect output 3 to an external phaser pedal, then connect
the output of the phaser to input 3. Then on Digit, connect a delay
to output 3, and then connect input 3 directly to output 1. The repeats
of that delay will now be running through the phaser. Be careful not
to create feedback paths, or route dry in parallel where some signal
goes through external pedals. The extra latency / phase difference
between the direct and effected signals can cause weird effects. So
it’s always safer just to effect things like delays / reverb tails.

\subsection{Rhythmic delays}

Tap a delay to see the details.
\begin{center}
\includegraphics[width=0.9\textwidth]{\string"D:/photos/Digit_screenshots/design/2020-06-26 17_42_52-Digit 2\string".png}
\par\end{center}

There are two modes. If you choose beats you can set a BPM and then
choose a subdivision in the right drop down menu. 
\begin{center}
\includegraphics[width=0.9\textwidth]{\string"D:/photos/Digit_screenshots/design/2020-06-26 17_43_25-Digit 2\string".png}
\par\end{center}

You can also set time in milliseconds, tap the time button and then
you can type in a number if you press the milliseconds field, or just
slide the slider to an approximate position.
\begin{center}
\includegraphics[width=0.9\textwidth]{\string"D:/photos/Digit_screenshots/design/2020-06-26 17_43_36-Digit 2\string".png}
\par\end{center}

The amount of milliseconds won't follow your tap tempo, so if you
have a tempo input plugged in the times here won't be relevant and
you should use the beat setting.

\subsection{Import reverbs / cabs}

To import new cab or reverb IRs onto Digit, copy them onto a USB flash
drive in folders called reverbs and cabs for each. Then go to settings
and tap Copy IRs. The files must be 48 kHz wavs. 

\subsection{Import / Export Presets}

\section{Digit Modules}

\subsection{Amp Bass Svt40 }

\subsection{Attenuverter }

The attenuverter attenuates and/or inverts a control signal. It has
two inputs which can be either be controlled by the sliders or by
a control input. So you can attenuate / invert one control signal
with another giving you ring mod / AM style control possibilities.
Parameters: output = multiplicand {*} multiplier.

\subsection{Auto Swell }

\subsection{Delay}
\begin{center}
\includegraphics[width=0.9\textwidth]{\string"D:/photos/Digit_screenshots/design/2020-06-26 17_26_07-Digit 2\string".png}
\par\end{center}

This is a delay with controllable read and write heads. Attach a footswitch
module to use tap tempo.

\subsubsection{parameters}
\begin{description}
\item [{Tme:}] delay time in beats. 
\item [{Bpm:}] tempo in beats per minute. Connect a footswitch module to
this to use tap tempo. 
\item [{Feedback:}] number of repeats 
\item [{Tone:}] blends between a tape style and digital style delay 
\item [{Level:}] level of the repeats 
\item [{Warp:}] the read head of the delay. Modulate this to get warble. 
\item [{Glide:}] how slowly a time change occurs. Effects warp and time. 
\end{description}

\subsection{Env Follower }

\subsection{Filter }

\subsection{Foot Switch A / B / C }

Connect level out to any control input or BPM out to any BPM input. 

\subsection{Freeze }

Holds what you're playing when the control level is active (1), creating
a drone.

\subsection{LFO }
\begin{center}
\includegraphics[width=0.9\textwidth]{\string"D:/photos/Digit_screenshots/design/2020-06-26 17_55_25-Digit 2\string".png}
\par\end{center}

Low Frequency Oscillator, this sends a control signal out and can
modulate other parameters. 

\subsection{Mix VCA }

\subsection{Mono EQ / Stereo EQ }

A power parametric EQ, available in mono or stereo versions. 

\subsection{Mono Cab}

Load a cab IR. 

\subsection{Mono Reverb / Stereo Reverb}

Convolution based reverb. The mono version loads a mono IR file, 1
in 1 out. The stereo version loads a stereo IR, 1 in 2 out. 

\subsection{Pan }

\subsection{Power Amp Cream}

\subsection{Power Amp Super}

\subsection{Reverse}

Reverse the incoming sound, by dividing it up into fragments. Try
very long or very short fragment lengths. 

\subsection{Saturator }

A tape emulating compressor. Push input gain up for more effect and
then balance level. 

\subsection{Slew Limiter}

Changes how fast a control signal changes. Connect it up to things
like foot switches to change them from being instant to slowly swelling. 

\subsection{Quad IR Cab / Quad IR Reverb}

Loads a quad channel IR, 2 in 2 out. Most likely not what you want.
Don’t use this unless you’ve got the right IR files and you’re sure
you need it. 

\subsection{Turntable Stop}

Emultates turning off a turntable. Works well connected to a foot
switch. 

\subsection{Warmth}

This blends between tape and tube style warmth/overdrive. 

\section{Regulatory / Licencing}

This device complies with Part 15 of FCC rules. Operation is subject
to the following two conditions: (1) This device may not cause harmful
interference, and (2) this device must accept any interference received,
including interference that may cause undesired operation

Digit includes code and art (such as fonts) under a variety of licenses
including GPL / LGPL / MIT / BSD / OFL. More details of this are available
here: \url{https://github.com/polyeffects/} 

\section{Thanks}

A giant collection of people have helped Poly Effects and Digit specifically.
These include Helen Davison (my mum), Claire Jeddou, Celeste Reno,
Jo Gardiner, David Robillard, Michelle Lam, Bernie Tschirren, Lisa
Bryant, Ed Pettersen, Josh Smith, Jordan Rudess, Filipe Coelho and
Robin Gareus. 
\end{document}
