%% LyX 2.3.4.2 created this file.  For more info, see http://www.lyx.org/.
%% Do not edit unless you really know what you are doing.
\documentclass{scrartcl}
\usepackage{fontspec}
\setmainfont[Ligatures=TeX,Numbers=OldStyle]{Barlow}
\setsansfont[Ligatures=TeX]{Barlow ExtraBold}
\setmonofont{Barlow Medium}
\usepackage[a4paper]{geometry}
\geometry{verbose,tmargin=2cm,bmargin=3cm,lmargin=2.5cm,rmargin=2.5cm,footskip=2cm}
\pagestyle{plain}
\setcounter{secnumdepth}{2}
\setcounter{tocdepth}{2}
\setlength{\parskip}{\medskipamount}
\setlength{\parindent}{0pt}
\usepackage{color}
\usepackage{url}
\usepackage{graphicx}
\usepackage{microtype}
\usepackage[unicode=true,
 bookmarks=true,bookmarksnumbered=false,bookmarksopen=true,bookmarksopenlevel=1,
 breaklinks=false,pdfborder={0 0 1},backref=false,colorlinks=true]
 {hyperref}
\hypersetup{pdftitle={Digit and Beebo Manual},
 pdfauthor={Poly Effects},
 pdfsubject={Digit and Beebo Manual},
 pdfkeywords={digit beebo},
 linkcolor=pink,citecolor=pink}

\makeatletter
\@ifundefined{date}{}{\date{}}
%%%%%%%%%%%%%%%%%%%%%%%%%%%%%% User specified LaTeX commands.
\addtokomafont{section}{\MakeUppercase}
\addtokomafont{subsection}{\MakeUppercase}
\usepackage{fontspec, newunicodechar}
\newfontfamily{\fallbackfont}{FreeSans}[Scale=MatchLowercase]
\DeclareTextFontCommand{\textfallback}{\fallbackfont}
\newunicodechar{¼}{\textfallback{¼}}
\newunicodechar{⅓}{\textfallback{⅓}}
\newunicodechar{⅛}{\textfallback{⅛}}
\definecolor{pink}{RGB}{255, 117, 208}

\makeatother

\begin{document}
\begin{center}
\includegraphics[width=0.9\columnwidth]{manual_images/FB-Cover-image-4-828x465px}
\par\end{center}

If you're ever confused by anything or need help please contact me.
This manual is a compliment to the videos. Please watch the videos
at \url{https://www.youtube.com/channel/UCwQ9E6imYd0qFPMEuEZjcdw}
\begin{quote}
Manual is always being improved, please contact me with any suggestions.

\newpage{}
\end{quote}
\tableofcontents{}

\newpage{}

\part{Gettings started}

\section{Start here }

\subsection{Power}

The first thing to do is power the pedal. It's 9V center negative
and requires at least 500mA. After that it'll start up. Start up time
is currently a bit long, so don't panic if nothing is happening for
a few seconds.

Once Digit / Beebo has started up we're at the main screen, with the
default preset (the first one in your set list) loaded. From here
we can add modules, change the settings of existing modules, connect
modules or load a preset. This leads to the question, what is a module?
\begin{center}
\includegraphics[width=0.9\textwidth]{\string"manual_images/design/2020-06-26 17_23_18-Digit 2\string".png}
\par\end{center}

\subsection{Modules}

Modules are the basic blocks you connect. Think of them as individual
effect pedals on a pedal board that you place and connect. There are
some modules that process audio, and some modules that you can use
to control other modules. The ability to control modules is the powerful
modular / eurorack style workflow that differentiates Digit / Beebo
from many other systems.

\subsection{Add a module}

To start, we'll add a delay to the default preset. To do this, tap
plus and choose delay. 
\begin{center}
\includegraphics[width=0.9\textwidth]{\string"manual_images/design/2020-06-26 17_24_12-Digit 2\string".png}
\par\end{center}

The new delay will appear on the screen and you can now drag it to
a comfortable spot. You can drag any module at any time to move it.
\begin{center}
\includegraphics[width=0.9\textwidth]{\string"manual_images/design/2020-06-26 17_24_43-Digit 2\string".png}
\par\end{center}

To start I recommend tapping the help button (the ? mark) which will
turn on the extra help labels. Remember that single modules can be
very powerful, so start with a small number of them and understand
them before adding more. It'll be much easier to work out what is
going on. The standard delay module has many features compared to
standalone delay pedals!

\subsection{Connect a module}

You'll now want to connect up the delay. Hold input 1, then with another
finger tap the delay you've added. This multi touch stuff is really
much easier to see on a video. Any recent video should show this connection
style.

This is just the same as connecting up pedals on a pedal board, but
you can split and merge signals easily. You choose a source and then
a module to connect to. So now hold delay 1 and then tap output 1
to connect them. If you're connecting things with multiple inputs
or outputs a screen will appear where you can select which to use. 
\begin{center}
\includegraphics[width=0.9\textwidth]{\string"manual_images/design/2020-06-26 17_25_35-Digit 2\string".png}
\par\end{center}

\subsection{Change settings }

We can now change the settings of the delay. Tap on it to get up the
controls. Many modules have extra controls in the side menus.
\begin{center}
\includegraphics[width=0.9\textwidth]{\string"manual_images/design/2020-06-26 17_26_07-Digit 2\string".png}
\par\end{center}

\subsection{Knobs}

Once you touch any slider to change it, it automatically maps to the
knobs. The right knob changes the value a lot (coarse) and the left
by a little (fine).

\subsection{Presets}

To save or load a preset tap the floppy disk icon. You'll also see
a set list option. This is what order you move through the presets
in your set list when pressing both center and right or center and
left foot switches together.

\subsection{Toggle Effects with Foot switches}

By default, the right most foot switch is bypass and the two others
don't do anything. Foot switches can act as tempo sources or as control
sources. 

If you just want to enable or disable something with a foot switch,
hold the module you want to control and the press the physical foot
switch you want to use. A small letter in a circle will appear showing
the foot switch you choose to use. You can remap all 3 foot switches
like this. You can also assign a single foot switch to multiple modules. 

If you want to toggle between to delays for example, set one to disabled
then assign both to the same foot switch, they will now toggle which
is active.

\subsection{Foot Switches as CV or Tempo Sources}

If you want to control a specific parameter with a foot switch, tap
add (+) and then select foot switch A or B. A is the left one, B is
the center, C is the right.
\begin{center}
\includegraphics[width=0.9\textwidth]{\string"manual_images/design/2020-06-26 17_33_59-Digit 2\string".png}
\par\end{center}

Now tap the foot switch module you've added and then tap connect.
You can now connect it to things you want to control. There are two
outputs, the tap tempo and its value. Value is if it's up or pressed
down. To connect tap tempo from the foot switch, hold the foot swtich
and tap the delay. Then choose the BPM out.
\begin{center}
\includegraphics[width=0.9\textwidth]{\string"manual_images/design/2020-06-26 17_34_26-Digit 2\string".png}
\par\end{center}

\begin{center}
\includegraphics[width=0.9\textwidth]{\string"manual_images/design/2020-06-26 17_35_03-Digit 2\string".png}
\par\end{center}

\subsection{LFO / control signals}

As you saw with foot switches, one module can control another. If
we now add a low frequency oscillator (LFO) we can use that to control
other settings on the delay. Add the LFO then hold it and tap the
delay. Then choose warp in the screen that appears.
\begin{center}
\includegraphics[width=0.9\textwidth]{\string"manual_images/design/2020-06-26 17_34_43-Digit 2\string".png}
\par\end{center}

Control signals are coloured green.
\begin{center}
\includegraphics[width=0.9\textwidth]{\string"manual_images/design/2020-06-26 17_36_08-Digit 2\string".png}
\par\end{center}

The effect will be quite strong, so if we want to reduce we can tap
on the LFO and reduce the level. A uni polar LFO will only got from
0-1 (positive) but if you set that to zero it'll go negative and positive.
\begin{center}
\includegraphics[width=0.9\textwidth]{\string"manual_images/design/2020-06-26 17_55_25-Digit 2\string".png}
\par\end{center}

We can also add other modules in between such as the attenuverter
that act on the control signals. We can invert the phase or reduce
the level with this as well, in case we want the same LFO to control
two different things at different amounts or phase.
\begin{center}
\includegraphics[width=0.9\textwidth]{\string"manual_images/design/2020-06-26 17_36_50-Digit 2\string".png}
\par\end{center}

The attenuverter attenuates and/or inverts a signal. It has two inputs
which can be either be controlled by the sliders or by a control input.
You can attenuate / invert one control signal with another allowing
you ring mod / AM style control possibilities.

\subsection{Firmware update}

If you have a very early Digit and currently have version 1.something
(eg 1.7) firmware installed (check in the setting menu) please contact
me before updating. 

For Beebo or January 2020 or later versions of Digit, please grab
the packages in the 'upgrade\_from\_2' folder from here and put them
on a usb flash drive and click upgrade. It should be very quick. It'll
restart itself. The packages need to all be on the USB drive and not
in a zip. The current packages are frontend, modules, ingen and extra-content.
If you've updated recently and know that one file hasn't changed then
you don't need to include it on the USB flash drive next update and
it'll update a little faster.

The USB flash drive must be formatted FAT32 and have one partition
on it. If the update fails, please try pressing export presets in
the settings screen. If that also fails then the USB flash drive is
unreadable, either due to having no partitions (partitonless drive)
or being the wrong format. Please contact me for help. Please use
a normal USB flash drive and not an external hard drive. The cheap
NXT ones from Staples work well. All Sandisk ones tested also worked
well. Verbatim ones tested have failed.

\subsection{Inputs as effects loops}

You can use the input / outputs of Digit as effect loops. For example,
physically connect output 3 to an external phaser pedal, then connect
the output of the phaser to input 3. Then on Digit, connect a delay
to output 3, and then connect input 3 directly to output 1. The repeats
of that delay will now be running through the phaser. Be careful not
to create feedback paths, or route dry in parallel where some signal
goes through external pedals. The extra latency / phase difference
between the direct and effected signals can cause weird effects. So
it’s always safer just to effect things like delays / reverb tails.

\subsection{Rhythmic delays}

Tap a delay to see the details.
\begin{center}
\includegraphics[width=0.9\textwidth]{\string"manual_images/design/2020-06-26 17_42_52-Digit 2\string".png}
\par\end{center}

There are two modes. If you choose beats you can set a BPM and then
choose a subdivision in the right drop down menu. 
\begin{center}
\includegraphics[width=0.9\textwidth]{\string"manual_images/design/2020-06-26 17_43_25-Digit 2\string".png}
\par\end{center}

You can also set time in milliseconds, tap the time button and then
you can type in a number if you press the milliseconds field, or just
slide the slider to an approximate position.
\begin{center}
\includegraphics[width=0.9\textwidth]{\string"manual_images/design/2020-06-26 17_43_36-Digit 2\string".png}
\par\end{center}

The amount of milliseconds won't follow your tap tempo, so if you
have a tempo input plugged in the times here won't be relevant and
you should use the beat setting.

\subsection{Import reverbs / cabs}

To import new cab or reverb IRs onto Digit, copy them onto a USB flash
drive in folders called reverbs and cabs for each. Then go to settings
and tap Copy IRs. The files must be 48 kHz wavs. 

\subsection{Import / Export Presets}

Press the floppy disk icon to get to the preset screen. Then press
export preset where it'll ask you if you want to export the current
preset or all presets.

\section{MIDI}

\subsection{Changing Presets}

To change a preset with a program message, press the floppy disk icon
to get to the preset screen. Tap set list and add preset here. The
number next to them is what program change message you should send
to change to that preset. This is also the order the foot switch pairs
with D and E will step through them. 

\subsection{CC Controls}

To learn a control, press the MIDI icon next to the slider. It will
then pulse until it detects a CC.

\section{Regulatory / Licencing}

This device complies with Part 15 of FCC rules. Operation is subject
to the following two conditions: (1) This device may not cause harmful
interference, and (2) this device must accept any interference received,
including interference that may cause undesired operation

Digit includes code and art (such as fonts) under a variety of licenses
including GPL / LGPL / MIT / BSD / OFL. More details of this are available
here: \url{https://github.com/polyeffects/} 

\section{Thanks}

A giant collection of people have helped Poly Effects and Digit specifically.
These include Helen Davison (my mum), Claire Jeddou, Celeste Reno,
Jo Gardiner, David Robillard, Michelle Lam, Bernie Tschirren, Lisa
Bryant, Ed Pettersen, Josh Smith, Jordan Rudess, Leon Todd, Filipe
Coelho and Robin Gareus. 

\part{Modules}

\subsection{Ad Env Level}

An attack decay envelope generator with start and end levels. Works with a trigger.

Envelopes take in a  trigger and produce a CV that changes over time.
                  This version has just 2 stages, attack and decay. Connect triggers to the envelope and then connect the envelope output to the parameter you want to change over time.
                  This version adds a level to attack to and then decay to compared to the standard version.

\subsubsection{Inputs}
\begin{description}
\item [CV] Gate, Reset Level, Trigger
\end{description}

\subsubsection{Outputs}
\begin{description}
\item [CV] Envelope Out
\end{description}

\subsubsection{Controls}
\begin{itemize}
\item Attack Time
\item Attack To Level
\item Decay Time
\item Decay To Level
\item Initial Level
\end{itemize}

\subsection{Ad Envelope}

A attack decay envelope generator. Works with a trigger.

Envelopes take in a trigger and produce a CV that changes over time.
                  This version has just 2 stages, attack and decay. Connect triggers to the envelope and then connect the envelope output to the parameter you want to change over time.
                  This version attacks to 1 from 0.

\subsubsection{Inputs}
\begin{description}
\item [CV] Gate, Trigger
\end{description}

\subsubsection{Outputs}
\begin{description}
\item [CV] Envelope Out
\end{description}

\subsubsection{Controls}
\begin{itemize}
\item Attack Time
\item Decay Time
\end{itemize}

\subsection{Adsr}

Basic ADSR envelope generator. Takes in a gate.

Envelopes take in a  trigger and produce a CV that changes over time.
                  This version has 4 stages, attack, decay, sustain and release. Connect triggers to the envelope and then connect the envelope output to the parameter you want to change over time.
                  This is an important part of a synthesis but can also be used to create things like autoswells. Useful when connected to a VCA.

\subsubsection{Inputs}
\begin{description}
\item [CV] Driving Signal
\end{description}

\subsubsection{Outputs}
\begin{description}
\item [CV] Envelope Out
\end{description}

\subsubsection{Controls}
\begin{itemize}
\item Attack Time
\item Decay Time
\item Release Time
\item Sustain Level
\item Trigger Threshold
\end{itemize}

\subsection{Algo Reverb}

A special effects algorthmic reverb, featuring longer tails than the convolution reverb. Based Parasites Oliverb firmware of Mutable Intruments Clouds module. 

This eurorack based algorithmic reverb is designed to be modulated by control values.
                 It's great for special effects, and general weirdness. It isn't well suited to realistic sounds. It's capable of pitch shifting and self oscilliation.

Please see the \href{https://mqtthiqs.github.io/parasites/clouds.html}{original module manual} for more details.

This video is helpful: \url{https://youtu.be/apgmvYbG_zQ}.

\subsubsection{Inputs}
\begin{description}
\item [CV] Blend, Damp, Decay, Diffusion, Hold, Modulation Amount, Modulation Speed, Pitch, Reverse, Size, Trigger, Pre-Delay
\item [Audio] L In, R In
\end{description}

\subsubsection{Outputs}
\begin{description}
\item [Audio] L Out, R Out
\end{description}

\subsubsection{Controls}
\begin{itemize}
\item Blend
\item Damp
\item Decay
\item Diffusion
\item Hold
\item Modulation Amount
\item Modulation Speed
\item Pitch
\item Pre-Delay
\item Reverse
\item Size
\end{itemize}

\subsection{Amp Bass}

SVT40 bass amp sim

A bass amp emulation. The tone stack has sections that can be tuned. Run a cab after this.

\subsubsection{Inputs}
\begin{description}
\item [Audio] In
\end{description}

\subsubsection{Outputs}
\begin{description}
\item [Audio] Out
\end{description}

\subsubsection{Controls}
\begin{itemize}
\item Bass
\item Cabswitch
\item Highswitch
\item Lowswitch
\item Middle
\item Midswitch
\item Treble
\item Volume
\end{itemize}

\subsection{Attenuverter}

The attenuverter changes level and/or phase of a control signal. A x B

An attenuverter attenuates and inverts signals.
                  You can combine signals using it, for example to ring modulate two LFOs together, or just reduce the level of a CV

\subsubsection{Inputs}
\begin{description}
\item [CV] A, B
\end{description}

\subsubsection{Outputs}
\begin{description}
\item [CV] Product
\end{description}

\subsubsection{Controls}
\begin{itemize}
\item A
\item B
\end{itemize}

\subsection{Auto Swell}

Automatically swells volume to remove note attack

Auto swell is a similar effect to using your volume knob but instead it automatically detects when to swell.If you are looking for something more complex, you can combine an envelope follower and VCA to create your own

\subsubsection{Inputs}
\begin{description}
\item [CV] Threshold
\item [Audio] In
\end{description}

\subsubsection{Outputs}
\begin{description}
\item [Audio] Out
\end{description}

\subsubsection{Controls}
\begin{itemize}
\item Downtime
\item Threshold
\item Uptime
\end{itemize}

\subsection{Bitmangle}

Brutal bitmangler. Warning, can get very loud. Based on Mutable Instuments Warps Parasite.

The bit mangler isn't a normal bit crusher, instead it degredation and cross modulationthe bitwise AND is much louder than the bitwise XOR end, so I recommend a compressor or something aftwards to control levels.

Please see the \href{https://mqtthiqs.github.io/parasites/warps.html}{original module manual} for more details.

This video is helpful: \url{}.

\subsubsection{Inputs}
\begin{description}
\item [CV] Bit Cv, Input Amp 2 Cv, Input Amp Cv, Xor Vs And Cv
\item [Audio] Carrier, Modulator
\end{description}

\subsubsection{Outputs}
\begin{description}
\item [Audio] Aux, Out
\end{description}

\subsubsection{Controls}
\begin{itemize}
\item Amp Or Freq
\item Bits
\item Input Amplitude 2
\item Int Osc
\item Xor Vs And
\end{itemize}

\subsection{Chaos Controller}

Powerful repeatable randomness source. Based on Mutable Instruments Marbles module.

A source of random gates and voltages, which offers an extensive amount of (voltage) control on all the different flavors of randomness it produces. The module gives the musician many different ways of imposing structure on the random events generated by the module: synchronization to external clocks, control of the repetition or novelty of the generated material, quantization of the voltages, or randomization of gates or voltages generated by traditional sequencers.

Please see the \href{https://www.mutable-instruments.net/modules/marbles/manual/}{original module manual} for more details.

This video is helpful: \url{https://youtu.be/NkGkHuS69a0}.

\subsubsection{Inputs}
\begin{description}
\item [CV] Deja Vu Input, T Bias Input, T Clock Trigger Input, T Jitter Input, T Rate Input, X Bias Input, X Clock Input, X Spread Input, X Steps Input
\item [Tempo] Bpm T Clock
\end{description}

\subsubsection{Outputs}
\begin{description}
\item [CV] T1 Output, T2 Output, T3 Output, X1 Output, X2 Output, X3 Output, Y Output
\end{description}

\subsubsection{Controls}
\begin{itemize}
\item Deja Vu Length
\item Deja Vu Probablity
\item External
\item T Clock Bpm
\item T Deja Vu
\item T Gate Bias
\item T Jitter
\item T Mode
\item T Range
\item X Clock Source Internal
\item X Deja Vu
\item X Distribution Bias
\item X Mode
\item X Range
\item X Scale
\item X Smoothness Steps
\item X Spread
\item Y Distribution Bias
\item Y Divider
\item Y Smoothness
\item Y Spread
\end{itemize}

\subsection{Chorus D}

8 voice multi dimensional chorus



\subsubsection{Inputs}
\begin{description}
\item [Audio] In
\item [Tempo] Bpm
\end{description}

\subsubsection{Outputs}
\begin{description}
\item [Audio] Out L, Out R
\end{description}

\subsubsection{Controls}
\begin{itemize}
\item Bpm
\item Delay
\item Depth
\item Deviation
\end{itemize}

\subsection{Chorus D Ext}

8 voice multi dimensional chorus CV LFO



\subsubsection{Inputs}
\begin{description}
\item [CV] Lfo Cv
\item [Audio] In
\end{description}

\subsubsection{Outputs}
\begin{description}
\item [Audio] Out L, Out R
\end{description}

\subsubsection{Controls}
\begin{itemize}
\item Delay
\item Depth
\item Deviation
\end{itemize}

\subsection{Cv To Midi Cc}

convert control to MIDI CC



\subsubsection{Inputs}
\begin{description}
\item [CV] Cv In
\end{description}

\subsubsection{Outputs}
\begin{description}
\item [MIDI] Midi Out
\end{description}

\subsubsection{Controls}
\begin{itemize}
\item Cc Number
\item Channel
\item Resolution
\end{itemize}

\subsection{Cv To Trigger}

A Schmitt trigger that converts external analog CV to digital on / off trigger



\subsubsection{Inputs}
\begin{description}
\item [CV] Trigger
\end{description}

\subsubsection{Outputs}
\begin{description}
\item [CV] Out Trigger
\end{description}

\subsection{Dahdsr}

A delay attack hold decay sustain release envelope generator



\subsubsection{Inputs}
\begin{description}
\item [CV] Attack Time, Decay Time, Delay Time, Gate, Hold Time, Release Time, Sustain Level, Trigger
\end{description}

\subsubsection{Outputs}
\begin{description}
\item [CV] Envelope Out
\end{description}

\subsubsection{Controls}
\begin{itemize}
\item Attack Time
\item Decay Time
\item Delay Time
\item Hold Time
\item Release Time
\item Sustain Level
\end{itemize}

\subsection{Delay}

Flexible delay module. Add modules on the repeats for variations.



\subsubsection{Inputs}
\begin{description}
\item [CV] Feedback, Glide, Level, Time, Tone, Warp
\item [Audio] In
\item [Tempo] Bpm
\end{description}

\subsubsection{Outputs}
\begin{description}
\item [Audio] Out
\end{description}

\subsubsection{Controls}
\begin{itemize}
\item Bpm
\item Enabled
\item Feedback
\item Glide
\item Level
\item Time
\item Tone
\item Warp
\end{itemize}

\subsection{Difference}

a - b for control signals



\subsubsection{Inputs}
\begin{description}
\item [CV] A Cv, B Cv
\end{description}

\subsubsection{Outputs}
\begin{description}
\item [CV] Output
\end{description}

\subsubsection{Controls}
\begin{itemize}
\item A
\item B
\end{itemize}

\subsection{Diode Ladder Lpf}

A diode ladder low pass filter similar to the vintage Japanese designs



\subsubsection{Inputs}
\begin{description}
\item [CV] Cutoff Cv, Q Cv
\item [Audio] In
\end{description}

\subsubsection{Outputs}
\begin{description}
\item [Audio] Out
\end{description}

\subsubsection{Controls}
\begin{itemize}
\item Cutoff
\item Q
\end{itemize}

\subsection{Doppler Panner}

binaural panner, allows positioning in 3D. Based on Parasite firware of Warps by Mutable Instruments.



Please see the \href{https://mqtthiqs.github.io/parasites/warps.html}{original module manual} for more details.

This video is helpful: \url{https://youtu.be/baHiSGgszQ4}.

\subsubsection{Inputs}
\begin{description}
\item [CV] Lfo Frequency Cv, Lfo Amplitude Cv, X Coordinate Cv, Y Coordinate Cv
\item [Audio] Left In, Right In
\end{description}

\subsubsection{Outputs}
\begin{description}
\item [Audio] Left Out, Right Out
\end{description}

\subsubsection{Controls}
\begin{itemize}
\item Lfo Amplitude
\item Lfo Frequency
\item Space Size
\item X Coordinate
\item Y Coordinate
\end{itemize}

\subsection{Drum Patterns}

Drum trigger explorer. Based on Grids by Mutable Instruments.



Please see the \href{https://mutable-instruments.net/modules/grids/manual/}{original module manual} for more details.

This video is helpful: \url{https://youtu.be/l5yN0N6aTws}.

\subsubsection{Inputs}
\begin{description}
\item [CV] Bd Fill Cv, Chaos Cv, Clock, Hh Fill Cv, Map X Cv, Map Y Cv, Reset, Run, Sn Fill Cv, Swing Cv
\item [Tempo] Bpm
\end{description}

\subsubsection{Outputs}
\begin{description}
\item [CV] Bass Drum Accent, Bass Drum Trigger, Hihat Accent, Hihat Trigger, Snare Accent, Snare Trigger
\end{description}

\subsubsection{Controls}
\begin{itemize}
\item Bass Drum Density
\item Chaos
\item Hihat Density
\item Map X
\item Map Y
\item Reset Button
\item Run Button
\item Snare Density
\item Swing
\item Tempo
\end{itemize}

\subsection{Env Follower}

Track an input signal and convert it into a control signal



\subsubsection{Inputs}
\begin{description}
\item [Audio] Audio In
\end{description}

\subsubsection{Outputs}
\begin{description}
\item [CV] Cv Out
\end{description}

\subsubsection{Controls}
\begin{itemize}
\item Attack Time
\item Decay Time
\item Invert
\item Maximum Value
\item Minimum Value
\item Peak/Rms
\item Saturation
\item Threshold
\end{itemize}

\subsection{Euclidean}

A euclidean sequencer with 4 tracks. Connect a trigger to go to the next step

The is the one outputs are triggered on the first beat and can be used to chain sequencers.

\subsubsection{Inputs}
\begin{description}
\item [CV] Back Trigger, Reset Trigger, Step Trigger
\end{description}

\subsubsection{Outputs}
\begin{description}
\item [CV] Is The One 1, Is The One 2, Is The One 3, Is The One 4, Trigger Out1, Trigger Out2, Trigger Out3, Trigger Out4
\item [Tempo] Tempo Out
\end{description}

\subsubsection{Controls}
\begin{itemize}
\item Loop 1 Beats
\item Loop 1 Is Enabled
\item Loop 1 Shift
\item Loop 1 Steps
\item Loop 2 Beats
\item Loop 2 Is Enabled
\item Loop 2 Shift
\item Loop 2 Steps
\item Loop 3 Beats
\item Loop 3 Is Enabled
\item Loop 3 Shift
\item Loop 3 Steps
\item Loop 4 Beats
\item Loop 4 Is Enabled
\item Loop 4 Shift
\item Loop 4 Steps
\end{itemize}

\subsection{Filter}

Virtual analog resonant low pass filter



\subsubsection{Inputs}
\begin{description}
\item [CV] Exp Fm, Fm, Resonance Mod
\item [Audio] Input
\end{description}

\subsubsection{Outputs}
\begin{description}
\item [Audio] Output
\end{description}

\subsubsection{Controls}
\begin{itemize}
\item Exp. Fm Gain
\item Frequency
\item Input Gain
\item Output Gain
\item Resonance
\item Resonance Gain
\end{itemize}

\subsection{Flanger}

flanger with internal LFO



\subsubsection{Inputs}
\begin{description}
\item [Audio] In
\item [Tempo] Bpm
\end{description}

\subsubsection{Outputs}
\begin{description}
\item [Audio] Out0
\end{description}

\subsubsection{Controls}
\begin{itemize}
\item Bpm
\item Delay
\item Depth
\item Feedback
\item Invert
\item Waveshape
\end{itemize}

\subsection{Flanger Ext}

flanger with CV for LFO



\subsubsection{Inputs}
\begin{description}
\item [CV] Lfo Cv
\item [Audio] In
\end{description}

\subsubsection{Outputs}
\begin{description}
\item [Audio] Out0
\end{description}

\subsubsection{Controls}
\begin{itemize}
\item Delay
\item Depth
\item Feedback
\item Invert
\end{itemize}

\subsection{Foot Switch A}

The left footswitch. Also available in Hector to allow patch compatability.



\subsubsection{Outputs}
\begin{description}
\item [CV] Value Out
\item [Tempo] Bpm Out
\end{description}

\subsubsection{Controls}
\begin{itemize}
\item Bpm
\item Is Latching
\item Off Value
\item On Value
\item Value
\item Value
\end{itemize}

\subsection{Foot Switch B}

The center footswitch. Also available in Hector to allow patch compatability.



\subsubsection{Outputs}
\begin{description}
\item [CV] Value Out
\item [Tempo] Bpm Out
\end{description}

\subsubsection{Controls}
\begin{itemize}
\item Bpm
\item Is Latching
\item Off Value
\item On Value
\item Value
\item Value
\end{itemize}

\subsection{Foot Switch C}

The right footswitch. Also available in Hector to allow patch compatability.



\subsubsection{Outputs}
\begin{description}
\item [CV] Value Out
\item [Tempo] Bpm Out
\end{description}

\subsubsection{Controls}
\begin{itemize}
\item Bpm
\item Is Latching
\item Off Value
\item On Value
\item Value
\item Value
\end{itemize}

\subsection{Foot Switch D}

The left and centre right footswitch. Also available in Hector to allow patch compatability.



\subsubsection{Outputs}
\begin{description}
\item [CV] Value Out
\item [Tempo] Bpm Out
\end{description}

\subsubsection{Controls}
\begin{itemize}
\item Bpm
\item Is Latching
\item Off Value
\item On Value
\item Value
\item Value
\end{itemize}

\subsection{Foot Switch E}

The right and centre footswitch. Also available in Hector to allow patch compatability.



\subsubsection{Outputs}
\begin{description}
\item [CV] Value Out
\item [Tempo] Bpm Out
\end{description}

\subsubsection{Controls}
\begin{itemize}
\item Bpm
\item Is Latching
\item Off Value
\item On Value
\item Value
\item Value
\end{itemize}

\subsection{Freeze}

Holds what you are playing when the control levelis active, creating a drone.



\subsubsection{Inputs}
\begin{description}
\item [CV] Freeze
\item [Audio] Audio In
\end{description}

\subsubsection{Outputs}
\begin{description}
\item [Audio] Audio Out
\end{description}

\subsubsection{Controls}
\begin{itemize}
\item Drone Gain
\item Freeze
\item Release
\end{itemize}

\subsection{Granular}

Granular texture generator, can work as a weird delay or reverb. Based on Parasite firmware of Mutable Instruments Clouds.



Please see the \href{https://www.mutable-instruments.net/modules/clouds/manual/}{original module manual} for more details.

This video is helpful: \url{https://youtu.be/g_Gue_MZ-Dk}.

\subsubsection{Inputs}
\begin{description}
\item [CV] Blend, Density, Freeze, Pitch, Position, Reverb, Reverse, Size, Spread, Texture, Trigger, Feedback
\item [Audio] L In, R In
\end{description}

\subsubsection{Outputs}
\begin{description}
\item [Audio] L Out, R Out
\end{description}

\subsubsection{Controls}
\begin{itemize}
\item Blend
\item Density
\item Feedback
\item Freeze
\item Pitch
\item Position
\item Reverb
\item Reverse
\item Size
\item Spread
\item Texture
\end{itemize}

\subsection{Harmonic Trem Ext}

harmonic tremolo CV LFO input



\subsubsection{Inputs}
\begin{description}
\item [CV] Lfo Cv
\item [Audio] In
\end{description}

\subsubsection{Outputs}
\begin{description}
\item [Audio] Out
\end{description}

\subsubsection{Controls}
\begin{itemize}
\item Crossoverfreq
\item Depth
\end{itemize}

\subsection{Harmonic Tremolo}

harmonic tremolo with internal LFO



\subsubsection{Inputs}
\begin{description}
\item [Audio] In
\item [Tempo] Bpm
\end{description}

\subsubsection{Outputs}
\begin{description}
\item [Audio] Out
\end{description}

\subsubsection{Controls}
\begin{itemize}
\item Bpm
\item Crossover Freq
\item Depth
\end{itemize}

\subsection{J Chorus}

chorus based on vintage Japanese synth chorus



\subsubsection{Inputs}
\begin{description}
\item [CV] 1 Enable, 2 Enable
\item [Audio] Audio Input 1, Audio Input 2
\end{description}

\subsubsection{Outputs}
\begin{description}
\item [Audio] Audio Output 1, Audio Output 2
\end{description}

\subsubsection{Controls}
\begin{itemize}
\item Chorus 1 Lfo Rate
\item Chorus 1 On/Off
\item Chorus 2 Lfo Rate
\item Chorus 2 On/Off
\end{itemize}

\subsection{K Org Hpf}

A high pass filter similar to the vintage Japanese designs



\subsubsection{Inputs}
\begin{description}
\item [CV] Cutoff Cv, Q Cv
\item [Audio] In
\end{description}

\subsubsection{Outputs}
\begin{description}
\item [Audio] Out
\end{description}

\subsubsection{Controls}
\begin{itemize}
\item Cutoff
\item Q
\end{itemize}

\subsection{K Org Lpf}

A low pass filter similar to the vintage Japanese designs



\subsubsection{Inputs}
\begin{description}
\item [CV] Cutoff Cv, Q Cv
\item [Audio] In
\end{description}

\subsubsection{Outputs}
\begin{description}
\item [Audio] Out
\end{description}

\subsubsection{Controls}
\begin{itemize}
\item Cutoff
\item Q
\end{itemize}

\subsection{Lfo}

Low frequency oscillator, send a control signal.



\subsubsection{Inputs}
\begin{description}
\item [CV] Reset
\item [Tempo] Bpm
\end{description}

\subsubsection{Outputs}
\begin{description}
\item [CV] Output
\end{description}

\subsubsection{Controls}
\begin{itemize}
\item Level
\item Shape Mod
\item Tempo
\item Tempo Multiplier
\item Unipolar
\item Wave Form
\end{itemize}

\subsection{Loop Common In}

Common input to the loopler



\subsubsection{Inputs}
\begin{description}
\item [Audio] Out
\end{description}

\subsection{Loop Common Out}

Common output from the loopler



\subsubsection{Outputs}
\begin{description}
\item [Audio] In
\end{description}

\subsection{Loop Extra Midi}

Connect internal MIDI generators to the Looper, do not connect external MIDI here, will duplicate.

For example trigger loops with a chaos controller module connected to a CV to CC then to this.

\subsubsection{Inputs}
\begin{description}
\item [MIDI] Out
\end{description}

\subsection{Looping Delay}

Granular pitch shifting, micro looping delay. Based on Parasite firmware of Clouds by Mutable Instruments.



Please see the \href{https://mqtthiqs.github.io/parasites/clouds.html}{original module manual} for more details.

This video is helpful: \url{https://youtu.be/6ltvGv43J3A}.

\subsubsection{Inputs}
\begin{description}
\item [CV] Blend, Diffusion, Filter, Loop, Pitch, Reverb, Reverse, Spread, Trigger, Feedback, Pitch Window, Tape Length
\item [Audio] L In, R In
\end{description}

\subsubsection{Outputs}
\begin{description}
\item [Audio] L Out, R Out
\end{description}

\subsubsection{Controls}
\begin{itemize}
\item Blend
\item Diffusion
\item Feedback
\item Filter
\item Freeze
\item Pitch
\item Pitch Windows
\item Reverb
\item Reverse
\item Spread
\item Tape Length
\end{itemize}

\subsection{Looping Envelope}

A powerful envelope and LFO generator.
Based on Tides by Mutable Instruments. 

Based on Tides by Mutable Instruments.

Please see the \href{https://www.mutable-instruments.net/modules/tides/manual/}{original module manual} for more details.

This video is helpful: \url{https://www.youtube.com/watch?v=SVfmMq_VcuI}.

\subsubsection{Inputs}
\begin{description}
\item [CV] Clock, Frequency, Shape, Shift, Slope, Smoothness, Trigger, V Per Oct
\end{description}

\subsubsection{Outputs}
\begin{description}
\item [CV] Out1, Out2, Out3, Out4
\end{description}

\subsubsection{Controls}
\begin{itemize}
\item Frequency
\item Frequency
\item Mode
\item Ramp
\item Range
\item Shape
\item Shape
\item Shift
\item Shift
\item Slope
\item Slope
\item Smoothness
\item Smoothness
\end{itemize}

\subsection{Macro Osc}

A powerful multi model oscillator voice. Based on Mutable Instruments Plaits module.



Please see the \href{https://www.mutable-instruments.net/modules/plaits/manual/}{original module manual} for more details.

This video is helpful: \url{https://youtu.be/_zYwdcYECdg}.

\subsubsection{Inputs}
\begin{description}
\item [CV] V Per Oct Cv, Engine Cv, Frequency Cv, Harmonics Cv, Level Cv, Morph Cv, Timbre Cv, Trigger Cv
\end{description}

\subsubsection{Outputs}
\begin{description}
\item [Audio] Aux, Out
\end{description}

\subsubsection{Controls}
\begin{itemize}
\item Frequency
\item Frequency Mod
\item Harmonics
\item Lpg Color
\item Lpg Decay
\item Model
\item Morph
\item Morph Mod
\item Timbre
\item Timbre Mod
\end{itemize}

\subsection{Max}

max of a, b also logical or



\subsubsection{Inputs}
\begin{description}
\item [CV] A Cv, B Cv
\end{description}

\subsubsection{Outputs}
\begin{description}
\item [CV] Output
\end{description}

\subsubsection{Controls}
\begin{itemize}
\item A
\item B
\end{itemize}

\subsection{Meta Modulation}

A powerful cross modulation module which applies an algorithm to the two inputs. Based on Warps by Mutable Instruments.



Please see the \href{https://www.mutable-instruments.net/modules/warps/manual/}{original module manual} for more details.

This video is helpful: \url{https://youtu.be/iRLU5B4V-Jw}.

\subsubsection{Inputs}
\begin{description}
\item [CV] Algorthm Cv, Level 1 Freq Cv, Level 2 Cv, Timbre Cv
\item [Audio] Carrier, Modulator
\end{description}

\subsubsection{Outputs}
\begin{description}
\item [Audio] Aux, Out
\end{description}

\subsubsection{Controls}
\begin{itemize}
\item Algorithm
\item Level 1 Or Freq
\item Level 2
\item Shape
\item Timbre
\end{itemize}

\subsection{Midi Cc}

MIDI CC to control value



\subsubsection{Inputs}
\begin{description}
\item [MIDI] Midi Input
\end{description}

\subsubsection{Outputs}
\begin{description}
\item [CV] Output
\end{description}

\subsubsection{Controls}
\begin{itemize}
\item Cc Number
\item Logarithmic
\item Maximum
\item Minimum
\end{itemize}

\subsection{Midi Clock In}

MIDI Clock to BPM



\subsubsection{Inputs}
\begin{description}
\item [MIDI] Midi Input
\end{description}

\subsubsection{Outputs}
\begin{description}
\item [Tempo] Bpm
\end{description}

\subsection{Midi Clock Out}

BPM to MIDI Clock



\subsubsection{Inputs}
\begin{description}
\item [Tempo] Bpm
\end{description}

\subsubsection{Outputs}
\begin{description}
\item [MIDI] Midi Out
\end{description}

\subsubsection{Controls}
\begin{itemize}
\item Bpm
\end{itemize}

\subsection{Midi Note To Cv}

convert MIDI notes to v per octave pitch CVs



\subsubsection{Inputs}
\begin{description}
\item [MIDI] Midi Input
\end{description}

\subsubsection{Outputs}
\begin{description}
\item [CV] Gate, Pitch, Velocity
\end{description}

\subsubsection{Controls}
\begin{itemize}
\item Cent
\item Channel
\item Octave
\item Panic
\item Retrigger
\item Semitone
\end{itemize}

\subsection{Min}

min of a, b also logical and



\subsubsection{Inputs}
\begin{description}
\item [CV] A Cv, B Cv
\end{description}

\subsubsection{Outputs}
\begin{description}
\item [CV] Output
\end{description}

\subsubsection{Controls}
\begin{itemize}
\item A
\item B
\end{itemize}

\subsection{Mono Eq}

Mono multiband parametric EQ



\subsubsection{Inputs}
\begin{description}
\item [Audio] In
\end{description}

\subsubsection{Outputs}
\begin{description}
\item [Audio] Out
\end{description}

\subsubsection{Controls}
\begin{itemize}
\item Bandwidth 1
\item Bandwidth 2
\item Bandwidth 3
\item Bandwidth 4
\item Enable
\item Frequency 1
\item Frequency 2
\item Frequency 3
\item Frequency 4
\item Gain
\item Gain 1
\item Gain 2
\item Gain 3
\item Gain 4
\item Highpass
\item Highpass Frequency
\item Highpass Resonance
\item Highshelf
\item Highshelf Bandwidth
\item Highshelf Frequency
\item Highshelf Gain
\item Lowpass
\item Lowpass Frequency
\item Lowpass Resonance
\item Lowshelf
\item Lowshelf Bandwidth
\item Lowshelf Frequency
\item Lowshelf Gain
\item Reset Peak Hold
\item Section 1
\item Section 2
\item Section 3
\item Section 4
\end{itemize}

\subsection{Mono Cab}

Mono cab sim



\subsubsection{Inputs}
\begin{description}
\item [CV] Output Gain
\item [Audio] In
\end{description}

\subsubsection{Outputs}
\begin{description}
\item [Audio] Out
\end{description}

\subsubsection{Controls}
\begin{itemize}
\item /Audio/Cabs/1X12Cab.Wav
\item Output Gain
\end{itemize}

\subsection{Mono Compressor}

RMS downward compressor with auto markup



\subsubsection{Inputs}
\begin{description}
\item [Audio] In
\end{description}

\subsubsection{Outputs}
\begin{description}
\item [Audio] Out
\end{description}

\subsubsection{Controls}
\begin{itemize}
\item Attack Time
\item Enable
\item Hold
\item Input Gain
\item Ratio
\item Release Time
\item Threshold
\end{itemize}

\subsection{Mono Reverb}

Mono convolution based reverb.



\subsubsection{Inputs}
\begin{description}
\item [CV] Output Gain
\item [Audio] In
\end{description}

\subsubsection{Outputs}
\begin{description}
\item [Audio] Out
\end{description}

\subsubsection{Controls}
\begin{itemize}
\item /Audio/Reverbs/Emt 140 Dark 1.Wav
\item Enabled
\item Output Gain
\end{itemize}

\subsection{Multi Resonator}

Resonator building block simulating multiple vibrating structures. Based on Rings by Mutable Instruments.

Please see the \href{https://www.mutable-instruments.net/modules/rings/manual/}{original module manual} for more details.

This video is helpful: \url{https://youtu.be/O0IHt1JiRvk}.

\subsubsection{Inputs}
\begin{description}
\item [CV] Brightness Mod, Damping Mod, Frequency Mod, Pitch, Position Mod, Structure Mod, Strum
\item [Audio] In
\end{description}

\subsubsection{Outputs}
\begin{description}
\item [Audio] Even, Odd
\end{description}

\subsubsection{Controls}
\begin{itemize}
\item Brightness
\item Brightness Mod
\item Damping
\item Damping Mod
\item Frequency
\item Frequency Mod
\item Internal Exicter
\item Polyphony
\item Position
\item Position Mod
\item Resonator
\item Structure
\item Structure Mod
\end{itemize}

\subsection{Onset Detect}

detects when a note starts and sends a trigger



\subsubsection{Inputs}
\begin{description}
\item [Audio] In
\end{description}

\subsubsection{Outputs}
\begin{description}
\item [CV] Gate
\end{description}

\subsubsection{Controls}
\begin{itemize}
\item Onset Threshold
\item Silence Threshold
\end{itemize}

\subsection{Oog Half Lpf}

A low pass filter inspired by vintage American designs



\subsubsection{Inputs}
\begin{description}
\item [CV] Cutoff Cv, Q Cv
\item [Audio] In
\end{description}

\subsubsection{Outputs}
\begin{description}
\item [Audio] Out
\end{description}

\subsubsection{Controls}
\begin{itemize}
\item Cutoff
\item Q
\end{itemize}

\subsection{Pan}

Control signal controlled panner



\subsubsection{Inputs}
\begin{description}
\item [CV] Pan Cv
\item [Audio] In
\end{description}

\subsubsection{Outputs}
\begin{description}
\item [Audio] Out L, Out R
\end{description}

\subsubsection{Controls}
\begin{itemize}
\item Pan Offset
\end{itemize}

\subsection{Phaser}

Basic phaser



\subsubsection{Inputs}
\begin{description}
\item [Audio] Audio Input 1
\end{description}

\subsubsection{Outputs}
\begin{description}
\item [Audio] Audio Output 1
\end{description}

\subsubsection{Controls}
\begin{itemize}
\item Color
\item Dry/Wet Mix
\item Feedback Bass Cut
\item Feedback Depth
\item Lfo Frequency
\end{itemize}

\subsection{Phaser Ext}

Mono phaser controlled by CV input



\subsubsection{Inputs}
\begin{description}
\item [CV] Lfo Cv
\item [Audio] Audio Input 1
\end{description}

\subsubsection{Outputs}
\begin{description}
\item [Audio] Audio Output 1
\end{description}

\subsubsection{Controls}
\begin{itemize}
\item Color
\item Dry/Wet Mix
\item Feedback Bass Cut
\item Feedback Depth
\end{itemize}

\subsection{Phaser Stereo Ext}

Stereo phaser controlled by CV input



\subsubsection{Inputs}
\begin{description}
\item [CV] Lfo Cv
\item [Audio] Audio Input 1, Audio Input 2
\end{description}

\subsubsection{Outputs}
\begin{description}
\item [Audio] Audio Output 1, Audio Output 2
\end{description}

\subsubsection{Controls}
\begin{itemize}
\item Color
\item Dry/Wet Mix
\item Feedback Bass Cut
\item Feedback Depth
\item Stereo Phase
\end{itemize}

\subsection{Pitch Detect}

BETA detects played notes and converts it volt per octave and MIDI



\subsubsection{Inputs}
\begin{description}
\item [Audio] In
\end{description}

\subsubsection{Outputs}
\begin{description}
\item [CV] Gate, V/Oct Pitch
\item [MIDI] Midi Out
\end{description}

\subsubsection{Controls}
\begin{itemize}
\item Onset Threshold
\item Pitch Detection Tolerance
\item Silence Threshold
\end{itemize}

\subsection{Pitch Shift}

a basic pitch shifter. It delays the input signal



\subsubsection{Inputs}
\begin{description}
\item [CV] V/Oct Pitch
\item [Audio] Input
\end{description}

\subsubsection{Outputs}
\begin{description}
\item [Audio] Output
\end{description}

\subsubsection{Controls}
\begin{itemize}
\item Dry Level
\item Semitone Shift
\item Wet Level
\end{itemize}

\subsection{Poly Note To Cv}

convert poly MIDI notes to v per octave pitch CVs



\subsubsection{Inputs}
\begin{description}
\item [MIDI] Midi Input
\end{description}

\subsubsection{Outputs}
\begin{description}
\item [CV] Gate 1, Gate 2, Gate 3, Gate 4, Pitch 1, Pitch 2, Pitch 3, Pitch 4, Velocity
\end{description}

\subsubsection{Controls}
\begin{itemize}
\item Cent
\item Channel
\item Octave
\item Panic
\item Semitone
\end{itemize}

\subsection{Power Amp Cream}

An attempt at a cream coloured power amp emulation.



\subsubsection{Inputs}
\begin{description}
\item [Audio] In
\end{description}

\subsubsection{Outputs}
\begin{description}
\item [Audio] Out
\end{description}

\subsubsection{Controls}
\begin{itemize}
\item Bass
\item Level
\item Treble
\item Volume
\end{itemize}

\subsection{Power Amp Super}

An attempt at a power amp emulation



\subsubsection{Inputs}
\begin{description}
\item [Audio] In
\end{description}

\subsubsection{Outputs}
\begin{description}
\item [Audio] Out
\end{description}

\subsubsection{Controls}
\begin{itemize}
\item Bass
\item Gain
\item Treble
\item Volume
\end{itemize}

\subsection{Product}

a times b for control signals



\subsubsection{Inputs}
\begin{description}
\item [CV] A Cv, B Cv
\end{description}

\subsubsection{Outputs}
\begin{description}
\item [CV] Output
\end{description}

\subsubsection{Controls}
\begin{itemize}
\item A
\item B
\end{itemize}

\subsection{Quad Ir Cab}

Requires quad channel IR. You do not want this unless you have special IRs



\subsubsection{Inputs}
\begin{description}
\item [CV] Gain
\item [Audio] Inl, Inr
\end{description}

\subsubsection{Outputs}
\begin{description}
\item [Audio] Outl, Outr
\end{description}

\subsubsection{Controls}
\begin{itemize}
\item /Audio/Cabs/1X12Cab.Wav
\item Gain
\end{itemize}

\subsection{Quad Ir Reverb}

Convolution reverb. Quad channel IRs required.



\subsubsection{Inputs}
\begin{description}
\item [CV] Output Gain
\item [Audio] Inl, Inr
\end{description}

\subsubsection{Outputs}
\begin{description}
\item [Audio] Outl, Outr
\end{description}

\subsubsection{Controls}
\begin{itemize}
\item /Audio/Reverbs/Emt 140 Dark 1.Wav
\item Enabled
\item Output Gain
\end{itemize}

\subsection{Quantizer}

quantize a v/oct signal to a musical scale.

Feed in an LFO to make it play specific notes instead of smoothly changing between pitches

\subsubsection{Inputs}
\begin{description}
\item [CV] Input
\end{description}

\subsubsection{Outputs}
\begin{description}
\item [CV] Changed, Output
\end{description}

\subsubsection{Controls}
\begin{itemize}
\item A
\item A#
\item B
\item C
\item C#
\item D
\item D#
\item E
\item F
\item F#
\item G
\item G#
\end{itemize}

\subsection{Ratio}

a divided by b for control signals



\subsubsection{Inputs}
\begin{description}
\item [CV] A Cv, B Cv
\end{description}

\subsubsection{Outputs}
\begin{description}
\item [CV] Output
\end{description}

\subsubsection{Controls}
\begin{itemize}
\item A
\item B
\end{itemize}

\subsection{Rectify Value}

rectify or get the absolute value of the input. 



\subsubsection{Inputs}
\begin{description}
\item [CV] A Cv
\end{description}

\subsubsection{Outputs}
\begin{description}
\item [CV] Output
\end{description}

\subsection{Resonestor}

dual voice four part resonator. Based on Parasite firmware of Clouds by Mutable Intstruments.



Please see the \href{https://mqtthiqs.github.io/parasites/clouds.html}{original module manual} for more details.

This video is helpful: \url{https://youtu.be/VOd5zx_WDyA}.

\subsubsection{Inputs}
\begin{description}
\item [CV] Chord, Decay, Filter, Harmonics, Pitch, Random Mod, Reverse, Scatter, Spread, Switch Voice, Timbre, Trigger
\item [Audio] L In, R In
\end{description}

\subsubsection{Outputs}
\begin{description}
\item [Audio] L Out, R Out
\end{description}

\subsubsection{Controls}
\begin{itemize}
\item Chord
\item Decay
\item Filter
\item Harmonics
\item Pitch
\item Random Mod
\item Reverse
\item Scatter
\item Spread
\item Switch Voice
\item Timbre
\end{itemize}

\subsection{Reverse}

Reverse effect. Try short fragment length for weird tremolo



\subsubsection{Inputs}
\begin{description}
\item [CV] Dry Level, Wet Level
\item [Audio] Input
\end{description}

\subsubsection{Outputs}
\begin{description}
\item [Audio] Output
\end{description}

\subsubsection{Controls}
\begin{itemize}
\item Dry Level
\item Fragment Length
\item Wet Level
\end{itemize}

\subsection{Rotary}

A rotating loudspeaker using physical modelling. Same sound as advanced.



\subsubsection{Inputs}
\begin{description}
\item [CV] Drum Brake, Drum Speed, Horn Brake, Horn Speed
\item [Audio] Input
\end{description}

\subsubsection{Outputs}
\begin{description}
\item [Audio] Left Output, Right Output
\end{description}

\subsubsection{Controls}
\begin{itemize}
\item Drum Level
\item Drum Stereo Width
\item Enable
\item Horn Level
\item Motors Ac/Dc
\end{itemize}

\subsection{Rotary Advanced}

A rotating loudspeaker using physical modelling. Same sound, more controls.



\subsubsection{Inputs}
\begin{description}
\item [CV] Drum Brake, Drum Speed, Horn Brake, Horn Speed
\item [Audio] Input
\end{description}

\subsubsection{Outputs}
\begin{description}
\item [Audio] Left Output, Right Output
\end{description}

\subsubsection{Controls}
\begin{itemize}
\item Drum Acceleration
\item Drum Brake Position
\item Drum Deceleration
\item Drum Filter Type
\item Drum Level
\item Drum Radius
\item Drum Speed Fast
\item Drum Speed Slow
\item Drum Stereo Width
\item Enable
\item Frequency
\item Frequency
\item Frequency
\item Gain
\item Gain
\item Gain
\item Horn Acceleration
\item Horn Brake Position
\item Horn Deceleration
\item Horn Filter-1 Type
\item Horn Filter-2 Type
\item Horn Level
\item Horn Radius
\item Horn Signal Leakage
\item Horn Speed Fast
\item Horn Speed Slow
\item Horn Stereo Width
\item Horn X-Axis Offset
\item Horn Z-Axis Offset
\item Link Speed Control
\item Microphone Angle
\item Microphone Distance
\item Motors Ac/Dc
\item Q
\item Q
\item Q
\end{itemize}

\subsection{Sample Hold}

sample and hold a CV value when a trigger goes high



\subsubsection{Inputs}
\begin{description}
\item [CV] Input, Trigger
\end{description}

\subsubsection{Outputs}
\begin{description}
\item [CV] Gate, Output
\end{description}

\subsubsection{Controls}
\begin{itemize}
\item Trigger Level
\end{itemize}

\subsection{Saturator}

Nonlinear saturation and soft limiting.



\subsubsection{Inputs}
\begin{description}
\item [CV] Postgain, Pregain
\item [Audio] Input
\end{description}

\subsubsection{Outputs}
\begin{description}
\item [Audio] Output
\end{description}

\subsubsection{Controls}
\begin{itemize}
\item Postgain
\item Pregain
\end{itemize}

\subsection{Slew Limiter}

Slows how fast a control signal changes. Useful with foot switches.



\subsubsection{Inputs}
\begin{description}
\item [CV] In
\end{description}

\subsubsection{Outputs}
\begin{description}
\item [CV] Out
\end{description}

\subsubsection{Controls}
\begin{itemize}
\item Time Down
\item Time Up
\end{itemize}

\subsection{Step Sequencer}

16 Step sequencer with internal clock



\subsubsection{Inputs}
\begin{description}
\item [CV] Back Gate, Play, Reset Trigger
\item [Tempo] Bpm
\end{description}

\subsubsection{Outputs}
\begin{description}
\item [CV] Value Out
\end{description}

\subsubsection{Controls}
\begin{itemize}
\item Back
\item Bpm
\item Glide
\item Loop Steps
\item Note Length
\item Play
\item Value 0
\item Value 1
\item Value 10
\item Value 11
\item Value 12
\item Value 13
\item Value 14
\item Value 15
\item Value 2
\item Value 3
\item Value 4
\item Value 5
\item Value 6
\item Value 7
\item Value 8
\item Value 9
\end{itemize}

\subsection{Step Sequencer Ext}

16 Step sequencer that takes in a trigger for next or previous



\subsubsection{Inputs}
\begin{description}
\item [CV] Back Trigger, Reset Trigger, Step Trigger
\end{description}

\subsubsection{Outputs}
\begin{description}
\item [CV] Value Out
\end{description}

\subsubsection{Controls}
\begin{itemize}
\item Glide
\item Loop Steps
\item Value 0
\item Value 1
\item Value 10
\item Value 11
\item Value 12
\item Value 13
\item Value 14
\item Value 15
\item Value 2
\item Value 3
\item Value 4
\item Value 5
\item Value 6
\item Value 7
\item Value 8
\item Value 9
\end{itemize}

\subsection{Stereo Eq}

Stereo multiband parametric EQ.



\subsubsection{Inputs}
\begin{description}
\item [Audio] In Left, In Right
\end{description}

\subsubsection{Outputs}
\begin{description}
\item [Audio] Out Left, Out Right
\end{description}

\subsubsection{Controls}
\begin{itemize}
\item Bandwidth 1
\item Bandwidth 2
\item Bandwidth 3
\item Bandwidth 4
\item Enable
\item Frequency 1
\item Frequency 2
\item Frequency 3
\item Frequency 4
\item Gain
\item Gain 1
\item Gain 2
\item Gain 3
\item Gain 4
\item Highpass
\item Highpass Frequency
\item Highpass Resonance
\item Highshelf
\item Highshelf Bandwidth
\item Highshelf Frequency
\item Highshelf Gain
\item Lowpass
\item Lowpass Frequency
\item Lowpass Resonance
\item Lowshelf
\item Lowshelf Bandwidth
\item Lowshelf Frequency
\item Lowshelf Gain
\item Section 1
\item Section 2
\item Section 3
\item Section 4
\end{itemize}

\subsection{Stereo Cab}

Stereo cab sim. You normally do not want this. Requires stereo IRs.



\subsubsection{Inputs}
\begin{description}
\item [CV] Output Gain
\item [Audio] In
\end{description}

\subsubsection{Outputs}
\begin{description}
\item [Audio] Outl, Outr
\end{description}

\subsubsection{Controls}
\begin{itemize}
\item /Audio/Cabs/1X12Cab.Wav
\item Output Gain
\end{itemize}

\subsection{Stereo Compress}

RMS downward compressor with auto markup



\subsubsection{Inputs}
\begin{description}
\item [Audio] In Left, In Right
\end{description}

\subsubsection{Outputs}
\begin{description}
\item [Audio] Out Left, Out Right
\end{description}

\subsubsection{Controls}
\begin{itemize}
\item Attack Time
\item Enable
\item Hold
\item Input Gain
\item Ratio
\item Release Time
\item Threshold
\end{itemize}

\subsection{Stereo Phaser}

Basic stereo phaser



\subsubsection{Inputs}
\begin{description}
\item [Audio] Audio Input 1, Audio Input 2
\end{description}

\subsubsection{Outputs}
\begin{description}
\item [Audio] Audio Output 1, Audio Output 2
\end{description}

\subsubsection{Controls}
\begin{itemize}
\item Color
\item Dry/Wet Mix
\item Feedback Bass Cut
\item Feedback Depth
\item Lfo Frequency
\item Stereo Phase
\end{itemize}

\subsection{Stereo Reverb}

Stereo convolution reverb



\subsubsection{Inputs}
\begin{description}
\item [CV] Output Gain
\item [Audio] In
\end{description}

\subsubsection{Outputs}
\begin{description}
\item [Audio] Outl, Outr
\end{description}

\subsubsection{Controls}
\begin{itemize}
\item /Audio/Reverbs/Emt 140 Dark 1.Wav
\item Enabled
\item Output Gain
\end{itemize}

\subsection{Sum}

a + b for control signals



\subsubsection{Inputs}
\begin{description}
\item [CV] A Cv, B Cv
\end{description}

\subsubsection{Outputs}
\begin{description}
\item [CV] Output
\end{description}

\subsubsection{Controls}
\begin{itemize}
\item A
\item B
\end{itemize}

\subsection{Tempo Ratio}

ratio between a tempo, like 3/4 to input tempo



\subsubsection{Inputs}
\begin{description}
\item [Tempo] Input Tempo
\end{description}

\subsubsection{Outputs}
\begin{description}
\item [Tempo] Output Tempo
\end{description}

\subsubsection{Controls}
\begin{itemize}
\item A
\item B
\end{itemize}

\subsection{Thruzero Flange}

Through Zero Flanger.



\subsubsection{Inputs}
\begin{description}
\item [CV] Depth, Depth Mod, Feedback, Mix, Rate
\item [Audio] Left In, Right In
\end{description}

\subsubsection{Outputs}
\begin{description}
\item [Audio] Left Out, Right Out
\end{description}

\subsubsection{Controls}
\begin{itemize}
\item Depth
\item Depth Mod
\item Feedback
\item Mix
\item Rate
\end{itemize}

\subsection{Time Stretch}

A granular time stretching and pitch shifting module. Based on Parasite firmware of Clouds by Mutable Instruments.



Please see the \href{https://mqtthiqs.github.io/parasites/clouds.html}{original module manual} for more details.

This video is helpful: \url{https://youtu.be/6ltvGv43J3A}.

\subsubsection{Inputs}
\begin{description}
\item [CV] Blend, Diffusion, Filter, Freeze, Pitch, Position, Reverb, Reverse, Size, Spread, Trigger, Feedback
\item [Audio] L In, R In
\end{description}

\subsubsection{Outputs}
\begin{description}
\item [Audio] L Out, R Out
\end{description}

\subsubsection{Controls}
\begin{itemize}
\item Blend
\item Diffusion
\item Feedback
\item Filter
\item Freeze
\item Pitch
\item Position
\item Reverb
\item Reverse
\item Size
\item Spread
\end{itemize}

\subsection{Toggle}

toggles the output value on every trigger



\subsubsection{Inputs}
\begin{description}
\item [CV] Trigger
\end{description}

\subsubsection{Outputs}
\begin{description}
\item [CV] Out Gate
\end{description}

\subsection{Turntable Stop}

Simulates turning off a turntable. Connect a control to pull the plug.



\subsubsection{Inputs}
\begin{description}
\item [CV] Decay Time, Pull The Plug
\item [Audio] Audio In
\end{description}

\subsubsection{Outputs}
\begin{description}
\item [Audio] Audio Out
\end{description}

\subsubsection{Controls}
\begin{itemize}
\item Decay Curve
\item Decay Time
\item Pull The Plug
\end{itemize}

\subsection{Twist Delay}

A delay where speed and length interact with quality. Based on Parasite firmware of Warps by Mutable Instruments.



Please see the \href{https://mqtthiqs.github.io/parasites/warps.html}{original module manual} for more details.

This video is helpful: \url{https://youtu.be/baHiSGgszQ4}.

\subsubsection{Inputs}
\begin{description}
\item [CV] Dry Wet Cv, Feedback Cv, Length Cv, Speed Direction Cv
\item [Audio] In 1, In 2
\end{description}

\subsubsection{Outputs}
\begin{description}
\item [Audio] Out 1, Out 2
\end{description}

\subsubsection{Controls}
\begin{itemize}
\item Dry Wet
\item Feedback
\item Length
\item Mode
\item Speed Direction
\end{itemize}

\subsection{Uberheim Filter}

A multi out filter inspired by vintage American designs



\subsubsection{Inputs}
\begin{description}
\item [CV] Cutoff Cv, Q Cv
\item [Audio] In
\end{description}

\subsubsection{Outputs}
\begin{description}
\item [Audio] Band Pass, Band Stop, High Pass, Low Pass
\end{description}

\subsubsection{Controls}
\begin{itemize}
\item Cutoff
\item Q
\end{itemize}

\subsection{Vca}

simple voltage controlled amplifier



\subsubsection{Inputs}
\begin{description}
\item [CV] Gain
\item [Audio] Input
\end{description}

\subsubsection{Outputs}
\begin{description}
\item [Audio] Output
\end{description}

\subsubsection{Controls}
\begin{itemize}
\item Gain
\end{itemize}

\subsection{Vibrato}

vibrato with internal LFO



\subsubsection{Inputs}
\begin{description}
\item [Audio] In
\item [Tempo] Bpm
\end{description}

\subsubsection{Outputs}
\begin{description}
\item [Audio] Out
\end{description}

\subsubsection{Controls}
\begin{itemize}
\item Bpm
\item Feedback Gain
\item Invert
\item Max Notch1 Freq
\item Min Notch1 Freq
\item Notch Depth
\item Notch Freq Ratio
\item Notch Width
\item Phase
\item Vibrato Mode
\end{itemize}

\subsection{Vibrato Ext}

vibrato with CV LFO



\subsubsection{Inputs}
\begin{description}
\item [CV] Lfo Cv
\item [Audio] In
\end{description}

\subsubsection{Outputs}
\begin{description}
\item [Audio] Out
\end{description}

\subsubsection{Controls}
\begin{itemize}
\item Feedback Gain
\item Invert
\item Max Notch1 Freq
\item Min Notch1 Freq
\item Notch Depth
\item Notch Freq Ratio
\item Notch Width
\item Vibrato Mode
\end{itemize}

\subsection{Warmth}

Tube triode emulation



\subsubsection{Inputs}
\begin{description}
\item [CV] Drive, Tape--Tube Blend
\item [Audio] Input
\end{description}

\subsubsection{Outputs}
\begin{description}
\item [Audio] Output
\end{description}

\subsubsection{Controls}
\begin{itemize}
\item Drive
\item Tape--Tube Blend
\end{itemize}

\subsection{Wavefolder}

Chebyshev wave folder. Based on Parasite firmware of Warps by Mutable Instruments. 



Please see the \href{https://mqtthiqs.github.io/parasites/warps.html}{original module manual} for more details.

This video is helpful: \url{https://youtu.be/baHiSGgszQ4}.

\subsubsection{Inputs}
\begin{description}
\item [CV] Input Amp 2 Cv, Input Amp Cv, Input Bias Cv, Num Fold Cv
\item [Audio] Carrier, Modulator
\end{description}

\subsubsection{Outputs}
\begin{description}
\item [Audio] Aux, Out
\end{description}

\subsubsection{Controls}
\begin{itemize}
\item Amp Or Freq
\item Input Amplitude 2
\item Input Bias
\item Int Osc
\item Number Of Folds
\end{itemize}

\subsection{Wet Dry}

blend between two inputs



\subsubsection{Inputs}
\begin{description}
\item [CV] Level, Wet Dry Blend
\item [Audio] Dry, Wet
\end{description}

\subsubsection{Outputs}
\begin{description}
\item [Audio] Out
\end{description}

\subsubsection{Controls}
\begin{itemize}
\item Level
\item Shape
\item Wet Dry
\end{itemize}

\subsection{Wet Dry Stereo}

blend stereo inputs to stereo out



\subsubsection{Inputs}
\begin{description}
\item [CV] Level, Wet Dry Blend
\item [Audio] Dry L, Dry R, Wet L, Wet R
\end{description}

\subsubsection{Outputs}
\begin{description}
\item [Audio] Out L, Out R
\end{description}

\subsubsection{Controls}
\begin{itemize}
\item Level
\item Shape
\item Signal A/B
\end{itemize}


\end{document}
