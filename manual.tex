%% LyX 2.3.4 created this file.  For more info, see http://www.lyx.org/.
%% Do not edit unless you really know what you are doing.
\documentclass{scrartcl}
\usepackage{fontspec}
\setmainfont[Ligatures=TeX,Numbers=OldStyle]{Metropolis}
\setsansfont[Ligatures=TeX]{Metropolis Extra Bold}
\setmonofont{Metropolis Medium}
\usepackage[a4paper]{geometry}
\geometry{verbose,tmargin=2cm,bmargin=3cm,lmargin=2.5cm,rmargin=2.5cm,footskip=2cm}
\pagestyle{plain}
\setcounter{secnumdepth}{2}
\setcounter{tocdepth}{2}
\setlength{\parskip}{\medskipamount}
\setlength{\parindent}{0pt}
\usepackage{color}
\usepackage{url}
\usepackage{graphicx}
\usepackage{microtype}
\usepackage[unicode=true,
 bookmarks=true,bookmarksnumbered=false,bookmarksopen=true,bookmarksopenlevel=1,
 breaklinks=false,pdfborder={0 0 1},backref=false,colorlinks=true]
 {hyperref}
\hypersetup{pdftitle={Digit Manual},
 pdfauthor={Poly Effects},
 pdfsubject={Poly Digit Manual},
 pdfkeywords={digt},
 linkcolor=pink,citecolor=pink}

\makeatletter
\@ifundefined{date}{}{\date{}}
%%%%%%%%%%%%%%%%%%%%%%%%%%%%%% User specified LaTeX commands.
\addtokomafont{section}{\MakeUppercase}
\addtokomafont{subsection}{\MakeUppercase}
\usepackage{fontspec, newunicodechar}
\newfontfamily{\fallbackfont}{Arial}[Scale=MatchLowercase]
\DeclareTextFontCommand{\textfallback}{\fallbackfont}
\newunicodechar{¼}{\textfallback{¼}}
\newunicodechar{⅓}{\textfallback{⅓}}
\newunicodechar{⅛}{\textfallback{⅛}}
\definecolor{pink}{RGB}{255, 117, 208}

\makeatother

\begin{document}
\begin{center}
\includegraphics[width=0.5\columnwidth]{\string"D:/photos/Digit_screenshots/DIGIT - Primary Colour Logo\string".pdf}
\par\end{center}

Thanks for making Digit by Poly Effects part of your board. We hope
you enjoy the sounds of Digit. We will continue to release more tutorial
videos and adding more help directly on the pedal. If you're ever
confused by anything or need help please contact us. This manual is
a compliment to the videos. Please watch the videos at \url{https://youtu.be/_1ZcEftfXt4}
\begin{center}
\includegraphics[width=0.8\textwidth]{D:/photos/Digit_screenshots/digit_product_720}
\par\end{center}
\begin{quote}
Manual is currently a work in progress!

\newpage{}
\end{quote}
\tableofcontents{}

\newpage{}

\section{Getting Started }

\subsection{Power}

The first thing to do is power Digit. It's 9V center negative and
requires 500mA. After that it'll start up. Start up time is currently
a bit long, so don't panic if nothing is happening for a few seconds.
This will be reduced in future firmware.

Once Digit has started up we're at the main screen, with the default
preset loaded. From here we can add modules, change the settings of
existing modules, connect modules or load a preset. This leads to
the question, what is a module?
\begin{center}
\includegraphics[width=0.9\textwidth]{\string"D:/photos/Digit_screenshots/screenshots/2020-02-10 19_25_14-Digit 2\string".png}
\par\end{center}

\subsection{Modules}

Modules are the basic blocks you connect in Digit. Think of them as
individual effect pedals on a pedal board that you place and connect.
There are some modules that process audio, and some modules that you
can use to control other modules. The ability to control modules is
the powerful modular / eurorack style workflow that differentiates
Digit from many other systems.

\subsection{Add a module}

To start, we'll add a delay to the default preset. To do this, tap
plus and choose delay. 

\includegraphics[width=0.9\textwidth]{\string"D:/photos/Digit_screenshots/screenshots/2020-02-10 19_27_37-Digit 2\string".png}

The new delay will appear on the screen and you can now drag it to
a comfortable spot. 
\begin{center}
\includegraphics[width=0.9\textwidth]{\string"D:/photos/Digit_screenshots/screenshots/2020-02-10 19_28_05-Digit 2\string".png}
\par\end{center}

One you're done moving it, exit move mode by tapping back. To start
I recommend tapping the help button (the ? mark) which will turn on
the extra help labels.

\subsection{Connect a module}

You'll now want to connect up the delay. Tap input 1, then tap connect. 

\includegraphics[width=0.9\textwidth]{\string"D:/photos/Digit_screenshots/screenshots/2020-02-10 19_28_38-Digit 2\string".png}

Now tap what you want to connect input 1 to, in this case the delay
1. This is just the same as connecting up pedals on a pedal board,
but you can split and merge signals easily. You're now in connect
mode, as you can see by the icon in the bottom middle. You choose
a source and then a module to connect to. So now tap delay 1 and then
tap output 1 to connect them. 

\includegraphics[width=0.9\textwidth]{\string"D:/photos/Digit_screenshots/screenshots/2020-02-10 19_28_48-Digit 2\string".png}

Now we can exit connect mode by tapping tapping back.

\includegraphics[width=0.9\textwidth]{\string"D:/photos/Digit_screenshots/screenshots/2020-02-10 19_28_57-Digit 2\string".png}

\subsection{Change settings }

We can now change the settings of the delay. Tap on it to get up the
controls. Many modules have special extra controls as well as this
overview. If they do you can tap the detail view (the eye icon) to
see them.

\includegraphics[width=0.9\textwidth]{\string"D:/photos/Digit_screenshots/screenshots/2020-02-10 19_30_12-Digit 2\string".png}

\subsection{Knobs}

Once you touch any slider to change it, it automatically maps to the
knobs. The right knob changes the value a lot (coarse) and the left
by a little (fine).

\subsection{Presets}

To save or load a preset tap the floppy disk icon. You'll also see
a set list option. This is what order you move through the presets
when pressing both center and right or center and left foot switches
together.

\subsection{Foot switches}

By default, the right most foot switch is bypass and the two others
don't do anything. Foot switches can act as tempo sources or as control
sources. To connect a foot switch to something, first tap add (+)
and then select foot switch A or B. A is the left one, B is the center. 

\includegraphics[width=0.9\textwidth]{\string"D:/photos/Digit_screenshots/screenshots/2020-02-10 19_33_25-Digit 2\string".png}

Now tap the foot switch module you've added and then tap connect.
You can now connect it to things you want to control. There are two
outputs, the tap tempo and its value. Value is if it's up or pressed
down. To connect tap tempo from the foot switch, choose the BPM out
and then tap delay 1.

\includegraphics[width=0.9\textwidth]{\string"D:/photos/Digit_screenshots/screenshots/2020-02-10 19_33_53-Digit 2\string".png}

\subsection{LFO / control signals}

As you saw with foot switches, one module can control another. If
we now add a low frequency oscillator (LFO) we can use that to control
other settings on the delay. Add the LFO 

\includegraphics[width=0.9\textwidth]{\string"D:/photos/Digit_screenshots/screenshots/2020-02-10 19_34_37-Digit 2\string".png}

then tap connect, and connect it to warp on delay 1.

\includegraphics[width=0.9\textwidth]{\string"D:/photos/Digit_screenshots/screenshots/2020-02-10 19_34_56-Digit 2\string".png}

Control signals are coloured green.

\includegraphics[width=0.9\textwidth]{\string"D:/photos/Digit_screenshots/screenshots/2020-02-10 19_35_01-Digit 2\string".png}

The effect will be quite strong, so if we want to reduce it we can
add other modules in between such as the attenuverter that act on
the control signals.

\includegraphics[width=0.9\textwidth]{\string"D:/photos/Digit_screenshots/screenshots/2020-02-10 19_36_59-Digit 2\string".png}

The attenuverter attenuates and/or inverts a signal. It has two inputs
which can be either be controlled by the sliders or by a control input.
You can attenuate / invert one control signal with another allowing
you ring mod / AM style control possibilities. In our case of wanting
to control the level or phase of the LFO, we’ll connect to the multiplicand
and then control the multiplier with the slider.

\includegraphics[width=0.9\textwidth]{\string"D:/photos/Digit_screenshots/screenshots/2020-02-10 19_37_13-Digit 2\string".png}

\includegraphics[width=0.9\textwidth]{\string"D:/photos/Digit_screenshots/screenshots/2020-02-10 19_38_27-Digit 2\string".png}

\subsection{Firmware update}

If you currently have version 1.something (eg 1.7) firmware installed
(check in the setting menu) get all the packages in the 'upgrade\_from\_1'
folder from here and put them on a usb flash drive. Tap upgrade firmware
and wait while nothing appears to happen for about 20 minutes. When
you press the firmware update button there's no feedback and nothing
happens at all, it is working but the early firmware update interface
is pretty terrible. Just walk off and come back in a bit. The pedal
should automatically restart with the new firmware. If it hasn't restarted
after 35 minutes, panic and contact me. It takes quite a while. Updates
in the 2 series only take a minute or two. 

The text that comes up at the end will have many messages and one
of the messages will say : 'Failed to unmount media.' This is okay.
Don’t turn it off then. It’ll will restart itself soon

A full video on how to copy the files and update is available here: 

\url{https://youtu.be/5fXnT4UiqGQ}

If you have version 2 installed, please grab just the 2 packages in
the 'upgrade\_from\_2' folder from here and put them on a usb flash
drive and click upgrade. It should be very quick. Restart when told
to.

The USB flash drive must be formatted FAT32 and have one partition
on it. If the update fails, please try pressing export presets in
the settings screen. If that also fails then the USB flash drive is
unreadable, either due to having no partitions (partitonless drive)
or being the wrong format. Please contact me for help. Please use
a normal USB flash drive and not an external hard drive. The cheap
NXT ones from Staples work well. All Sandisk ones tested also worked
well. Verbatim ones tested have failed.

\subsection{Inputs as effects loops}

You can use the input / outputs of Digit as effect loops. For example,
physically connect output 3 to an external phaser pedal, then connect
the output of the phaser to input 3. Then on Digit, connect a delay
to output 3, and then connect input 3 directly to output 1. The repeats
of that delay will now be running through the phaser. Be careful not
to create feedback paths, or route dry in parallel where some signal
goes through external pedals. The extra latency / phase difference
between the direct and effected signals can cause weird effects. So
it’s always safer just to effect things like delays / reverb tails.

\subsection{Rhythmic delays}

Tap a delay to get the overview, then tap to see the main controls
view (the eye icon).

\includegraphics[width=0.9\textwidth]{\string"D:/photos/Digit_screenshots/screenshots/2020-02-10 19_30_12-Digit 2\string".png}

The top slider is the number of ¼ notes. You can then choose under
it the extra subdivision you want. 

\includegraphics[width=0.9\textwidth]{\string"D:/photos/Digit_screenshots/screenshots/2020-02-10 19_30_28-Digit 2\string".png}

For example, if you choose zero for number of ¼ notes and then ⅓ you’ll
have the delay on the triplets. If you choose 4 ¼ notes and ⅓ it’ll
be on the first triplet of the second bar. You can also just set delay
times via milliseconds if you want. The preset quarter dot 8 is a
good starting point, showing a ¼ note delay and a dotted ⅛ delay.

\includegraphics[width=0.9\textwidth]{\string"D:/photos/Digit_screenshots/screenshots/2020-02-10 19_42_02-Digit 2\string".png}

You can see the total time in ms and in the slider the total beat
+ fraction. In this case 4.75 as we’ve got our number of beats set
to 4, and our subdivision set to ⅛ dotted, which is 0.75 of a ¼ note.

\subsection{Import reverbs / cabs}

To import new cab or reverb IRs onto Digit, copy them onto a USB flash
drive in folders called reverbs and cabs for each. Then go to settings
and tap Copy IRs. The files must be 48 kHz wavs. 

\subsection{Import / Export Presets}

\section{Digit Modules}

\subsection{Amp Bass Svt40 }

\subsection{Attenuverter }

The attenuverter attenuates and/or inverts a control signal. It has
two inputs which can be either be controlled by the sliders or by
a control input. So you can attenuate / invert one control signal
with another giving you ring mod / AM style control possibilities.
Parameters: output = multiplicand {*} multiplier.

\subsection{Auto Swell }

\subsection{Delay}

This is a delay with controllable read and write heads. Attach a footswitch
module to use tap tempo.

\subsubsection{parameters}
\begin{description}
\item [{Tme:}] delay time in beats. 
\item [{Bpm:}] tempo in beats per minute. Connect a footswitch module to
this to use tap tempo. 
\item [{Feedback:}] number of repeats 
\item [{Tone:}] blends between a tape style and digital style delay 
\item [{Level:}] level of the repeats 
\item [{Warp:}] the read head of the delay. Modulate this to get warble. 
\item [{Glide:}] how slowly a time change occurs. Effects warp and time. 
\end{description}

\subsection{Env Follower }

\subsection{Filter }

\subsection{Foot Switch A / B / C }

Connect level out to any control input or BPM out to any BPM input. 

\subsection{Freeze }

Holds what you're playing when the control level is active (1), creating
a drone.

\subsection{LFO }

Low Frequency Oscillator, this sends a control signal out and can
modulate other parameters. 

\subsection{Mix VCA }

\subsection{Mono EQ / Stereo EQ }

A power parametric EQ, available in mono or stereo versions. 

\subsection{Mono Cab}

Load a cab IR. 

\subsection{Mono Reverb / Stereo Reverb}

Convolution based reverb. The mono version loads a mono IR file, 1
in 1 out. The stereo version loads a stereo IR, 1 in 2 out. 

\subsection{Pan }

\subsection{Power Amp Cream}

\subsection{Power Amp Super}

\subsection{Reverse}

Reverse the incoming sound, by dividing it up into fragments. Try
very long or very short fragment lengths. 

\subsection{Saturator }

A tape emulating compressor. Push input gain up for more effect and
then balance level. 

\subsection{Slew Limiter}

Changes how fast a control signal changes. Connect it up to things
like foot switches to change them from being instant to slowly swelling. 

\subsection{Quad IR Cab / Quad IR Reverb}

Loads a quad channel IR, 2 in 2 out. Most likely not what you want.
Don’t use this unless you’ve got the right IR files and you’re sure
you need it. 

\subsection{Turntable Stop}

Emultates turning off a turntable. Works well connected to a foot
switch. 

\subsection{Warmth}

This blends between tape and tube style warmth/overdrive. 

\section{Regulatory / Licencing}

This device complies with Part 15 of FCC rules. Operation is subject
to the following two conditions: (1) This device may not cause harmful
interference, and (2) this device must accept any interference received,
including interference that may cause undesired operation

Digit includes code and art (such as fonts) under a variety of licenses
including GPL / LGPL / MIT / BSD / OFL. More details of this are available
here: \url{https://github.com/polyeffects/} 

\section{Thanks}

A giant collection of people have helped Poly Effects and Digit specifically.
These include Helen Davison (my mum), Claire Jeddou, Celeste Reno,
Jo Gardiner, David Robillard, Michelle Lam, Bernie Tschirren, Lisa
Bryant, Ed Pettersen, Josh Smith, Jordan Rudess, Filipe Coelho and
Robin Gareus. 
\end{document}
